\chapter{Interfaces} \label{interfaces}

\section{Global Signals} \label{global-signals}

The following common signals are shared between all devices on the APB4
bus.

\begin{longtable}[]{@{}lccl@{}}
	\toprule
	Port & Size & Direction & Description\tabularnewline
	\midrule
	\endhead
	\texttt{PRESETn} & 1 & Input & Asynchronous active low reset\tabularnewline
	\texttt{PCLK} & 1 & Input & Clock Input\tabularnewline
	\bottomrule
	\caption{Global Signals}
\end{longtable}


\subsection{PRESETn} \label{presetn}

When the active low asynchronous \texttt{PRESETn} input is asserted (`0'), the
APB4 interface is put into its initial reset state.

\subsection{PCLK} \label{pclk}

\texttt{PCLK} is the APB4 interface system clock. All internal logic for the APB4
interface operates at the rising edge of this system clock and APB4 bus
timings are related to the rising edge of \texttt{PCLK}.

\section{Master Interface} \label{master-interface-1}

The APB4 Interface decodes the signaling of an APB4 bus master and
therefore implements a subset of a regular APB4 Slave Interface.

\begin{longtable}[]{@{}lccl@{}}
	\toprule
  	Port         & Size        & Direction & Description\tabularnewline
	\midrule
	\endhead
	  \texttt{MST\_PSEL}    & 1           & Input  & Peripheral Select\tabularnewline
	  \texttt{MST\_PADDR}   & \texttt{PADDR\_SIZE} & Input  & Address Bus\tabularnewline
	  \texttt{MST\_PRDATA}  & \texttt{PDATA\_SIZE} & Output & Read Data Bus\tabularnewline
	  \texttt{MST\_PREADY}  & 1           & Output & Transfer Ready\tabularnewline
	  \texttt{MST\_PSLVERR} & 1           & Output & Transfer Error Indicator\tabularnewline
	\bottomrule
	\caption{APB4 Master Interface Ports}
\end{longtable}

\subsection{MST\_PSEL}\label{mst_psel}

%The APB4 Master generates PSEL, to signal to an attached peripheral that
it is selected and a data transfer is pending. This signal drives the
APB4 Multiplexer MST\_PSEL port and is decoded to select the individual
peripheral by asserting the corresponding \texttt{SLV\_PSEL[n]} output..

\subsection{MST\_PADDR}\label{mst_paddr}

\texttt{MST\_PADDR} is the APB4 address bus. The bus width is defined by the
\texttt{PADDR\_SIZE} parameter and is driven by the APB4 Master.

\subsection{MST\_PRDATA}\label{mst_prdata}

\texttt{MST\_PRDATA} drives the APB4 read data bus. The selected peripheral
drives this bus during read cycles, via the APB4 Multiplexer.

The bus width must be byte-aligned and is defined by the \texttt{PDATA\_SIZE}
parameter.

\subsection{MST\_PREADY}\label{mst_pready}

\texttt{MST\_PREADY} is driven by the selected peripheral via the APB4
Multiplexer. It is used to extend an APB4 transfer.

\subsection{MST\_PSLVERR}\label{mst_pslverr}

\texttt{MST\_PSLVERR} indicates a failed data transfer to the APB4 Master when
asserted (`1') and is driven by the selected peripheral via the APB4
Multiplexer.

\section{Slave Interface}\label{slave-interface}

The Slave Interface provides the following signals \emph{for each}
individual peripheral. The number of peripherals supported, and
therefore instances of the following signals, is controlled by the
SLAVES parameter (see section 0).

\begin{quote}
	
	\textbf{Note:} Each individual port name is referenced by the index `n', where `n' is
	an integer value in the range 0 to \texttt{SLAVES-1}. E.g. \texttt{SLV\_PSEL[2]}		
	This nomenclature is used throughout this datasheet
	
\end{quote}

\begin{longtable}[]{@{}lccl@{}}
	\toprule
		Port & Size & Direction & Description\tabularnewline
	\midrule
	\endhead
		\texttt{SLV\_PSEL[n]}    & 1           & Output & Peripheral Select\tabularnewline
		\texttt{SLV\_PRDATA[n]}  & \texttt{PDATA\_SIZE} & Input  & Read Data Bus\tabularnewline
		\texttt{SLV\_PREADY[n]}  & 1           & Input  & Transfer Ready Input\tabularnewline
		\texttt{SLV\_PSLVERR[n]} & 1           & Input  & Transfer Error Indicator\tabularnewline
		\texttt{SLV\_ADDR[n]}    & \texttt{PADDR\_SIZE} & Input  & Transfer Ready Input\tabularnewline
		\texttt{SLV\_MASK[n]}    & \texttt{PADDR\_SIZE} & Input  & Transfer Error Indicator\tabularnewline
	\bottomrule
	\caption{Slave Interface Ports}
\end{longtable}

\subsection{SLV\_PSEL[n]}\label{slv_pseln}

The APB4 Multiplexer generates \texttt{SLV\_PSEL[n]}, signaling to an
attached peripheral that it is selected and a data transfer is pending.

\subsection{SLV\_PRDATA[n]}\label{slv_prdatan}

\texttt{SLV\_PRDATA[n]} is the APB4 read data bus associated with the
attached peripheral. The peripheral drives this bus during read cycles,
indicated when \texttt{PWRITE} is negated (`0'), and the data is then multiplexed
to the \texttt{MST\_PRDATA} output port.

The bus width must be byte-aligned and is defined by the \texttt{PDATA\_SIZE}
parameter.

\subsection{SLV\_PREADY[n]}\label{slv_preadyn}

\texttt{SLV\_PREADY[n]} is driven by the attached peripheral and multiplexed
to the \texttt{MST\_PREADY }output port. It is used to extend an APB4 transfer.

\subsection{SLV\_PSLVERR[n]}\label{slv_pslverrn}

\texttt{SLV\_PSLVERR[n]} indicates a failed data transfer when asserted
(`1'). As APB4 peripherals are not required to support this signal it must be
tied LOW (`0') when unused.

\subsection{SLV\_ADDR[n] and SLV\_MASK[n]} \label{slv_addrn-and-slv_maskn}

\texttt{SLV\_ADDR[n]} is the base address where the attached peripheral is to
appear in the system memory map. It is bitwise `AND'ed with the
corresponding address mask \texttt{SLV\_MASK[n]} input to define the overall
address range of each peripheral.

As a consequence, these ports are typically assigned hard-coded values
rather than connected to other logic in the design.